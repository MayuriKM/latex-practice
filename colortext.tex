\documentclass{article}
\usepackage[utf8]{inputenc}
\usepackage[english]{babel}

\usepackage[usenames, dvipsnames]{color}

\definecolor{mypink1}{rgb}{0.858, 0.188, 0.478}
\definecolor{mypink2}{RGB}{219, 48, 122}
\definecolor{mypink3}{cmyk}{0, 0.7808, 0.4429, 0.1412}
\definecolor{mygray}{gray}{0.6}

\begin{document}

This example shows different examples on how to use the \texttt{color} package
to change the colour of elements in \LaTeX.

\begin{itemize}
\color{ForestGreen}
\item First item
\item Second item
\end{itemize}

\noindent
{\color{RubineRed} \rule{\linewidth}{0.5mm} }

The background colour of some text can also be \textcolor{red}{easily} set. For
instance, you can change to orange the background of \colorbox{BurntOrange}{this
text} and then continue typing.


User-defined colours with different colour models:

\begin{enumerate}
\item \textcolor{mypink1}{Pink with rgb}
\item \textcolor{mypink2}{Pink with RGB}
\item \textcolor{mypink3}{Pink with cmyk}
\item \textcolor{mygray}{Gray with gray}
\end{enumerate}

\end{document}
