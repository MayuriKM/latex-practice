\documentclass[a4paper, oneside, 12pt, parskip=half]{scrbook}

% TODO: moving to property
\usepackage{xcolor}
\usepackage{graphicx}
   
\usepackage[framemethod=TikZ]{mdframed}
%   Principal MPD frame type
\mdfdefinestyle{mpdframe}{
  frametitlebackgroundcolor   = black!5,
  frametitlerule              = true,
  roundcorner                 = 1pt,
  middlelinewidth             = 1pt,
  innermargin                 = 0.5cm,
  outermargin                 = 0.5cm,
  innerleftmargin             = 0.5cm,
  innerrightmargin            = 0.5cm,
  innertopmargin              = \topskip,
  innerbottommargin           = \topskip,
}
\mdfdefinestyle{box}{
  style       = mpdframe,
  frametitle  = {Title of Frame},
}
\newmdenv[style=box]{custombox}

\definecolor{mycolor}{HTML}{FF7F00}
\renewcommand{\contentsname}{Table of content}
% Hyper links.
\usepackage[hyperfootnotes=false]{hyperref}
\hypersetup{
    colorlinks=true,
    linkcolor=blue,
    filecolor=blue,
    urlcolor=blue,
}
\begin{document}

\tableofcontents

\chapter{Chapter 1 of TeXBook}

2 backticks and 2 single quotes surround a double-quoted string, like ``this''.

\section{Control Sequences}

2 types: \textit{control word} and \textit{control symbol}.

`\texttt{\textbackslash TeX}' (\TeX) is a \textit{control word} because it is made up of \textit{letters}. Any non-letter will end the sequence.

`\texttt{\textbackslash '}' (accent on the letter \texttt{o}: \'o) is a \textit{control symbol} because it is made up of a single \textit{non-letter}. The sequence length will always be 1 character (symbol).

To inject a space after a \textit{control word}, do a `\texttt{\textbackslash \textvisiblespace}'.

\section{Playing with \texttt{hbox}, \texttt{lower}, \texttt{kern}}

A\kern -.1667em\lower .5ex \hbox {B}\kern -.125emC

Reference at https://www.tug.org/utilities/plain/cseq.html.

\section{Playing with color}

\textbf{Start back color {\color{red}followed by a red fragment}, going black again.} \textit{{\color{green} here is green color section.}} \texttt{\color{mycolor} My custom color}

\section{Working with image}

\begin{figure}
\includegraphics[width=\textwidth]{latex.jpeg}
  \caption{Latex Logo.}
  \label{fig:Latex}
\end{figure}

\newpage
\section{Custom frame}

\begin{custombox}
My custom box with Title.
\end{custombox}

\end{document}
