\documentclass{scrbook}

\usepackage{xparse}
\usepackage{tcolorbox}
\tcbuselibrary{listings,skins,theorems,xparse}
\NewTotalTCBox{\inlinebox}{ O{white} v}
{ tcbox raise base,
  arc=2pt,
  nobeforeafter,% makes the box inline
  colback=#1!10!white,% pales given color for background
  boxsep=0pt,left=1pt,right=1pt,top=2pt,bottom=2pt,
  boxrule=0pt}
{\lstinline[style=tcblatex,texcsstyle=*\color{blue}\bfseries]§#2§}
\newcommand\cil{\inlinebox[green]}

\tcbset{
  codefullblock/.style={
    tcbox raise base,
    listing only,
    listing options={
      language=TeX, style=tcblatex, texcsstyle=*\color{blue}\bfseries
    },
    arc=5pt,
    boxsep=5pt,left=1pt,right=1pt,top=2pt,bottom=2pt,
    boxrule=0pt},
  outputstyle/.style={colback=gray!10},
  codestyle/.style={colback=green!10}
}

\newtcblisting{outputcfb}{
  codefullblock, outputstyle
}

\newtcblisting{codecfb}{
  codefullblock, codestyle
}

\newenvironment{nbox}
  {% Begin codes
    \begin{center}
    \begin{tabular}{|p{0.7\textwidth}|}
    \hline\\
  }
  {
    \\\\\hline
    \end{tabular}
    \end{center}
  }

  \newenvironment{nbox2}
    {% Begin codes
      \begin{tabular}{|p{0.99\textwidth}|}
      \hline\\
    }
    {
      \\\\\hline
      \end{tabular}
    }

\newenvironment{nboxt}[1]
  {% Begin codes
    \begin{center}
    \begin{tabular}{|p{0.7\textwidth}|}
    \hline\\
    \bff{#1}\\

  }
  {
    \\\\\hline
    \end{tabular}
    \end{center}
  }

  \NewTotalTCBox{\tunebox}{ O{white} v}
  { colupper=black,nobeforeafter,tcbox raise base,
    arc=5pt,outer arc=0pt,colback=#1!10!white,
    colframe=#1!50!black,
    boxsep=0pt,left=1pt,right=1pt,top=2pt,bottom=2pt,
    boxrule=0pt,bottomrule=1pt,toprule=1pt}
  {\lstinline[style=tcblatex,texcsstyle=*\color{pink}\bfseries]§#2§}


\begin{document}

\textbf{Example 1} \\
\begin{outputcfb}
  \ExplSyntaxOn
  \tl_set:Nn \l_first_tl { Always, }
  \tl_set:Nn \l_second_tl { ~hope~for~the~BEST. }
  \tl_put_right:Nn \l_first_tl { \l_second_tl }% two variables are join here

  \DeclareDocumentCommand \foo { } {%
    \tl_use:N \l_first_tl
  }
  \ExplSyntaxOff

  \foo % output will show contents of both variable join together
\end{outputcfb}

\ExplSyntaxOn
\tl_set:Nn \l_first_tl { Always, }
\tl_set:Nn \l_second_tl { ~hope~for~the~BEST. }
\tl_put_right:Nn \l_first_tl { \l_second_tl }% two variables are join here
\DeclareDocumentCommand \foo { } {%
  \tl_use:N \l_first_tl
}
\ExplSyntaxOff
\begin{nbox2}
\textbf{Output 1:} \\
\foo \\
\end{nbox2}

\linebreak

\textbf{Example 2} \\
\begin{outputcfb}
  \ExplSyntaxOn
  \tl_set:Nn \l_first_s_tl { Example }
  \tl_set:Nn \l_second_s_tl { ~of~"NV"}
  \tl_put_right:NV \l_first_s_tl\l_second_s_tl %correct way to pass the value of variable and not name by using a V-type.

  \DeclareDocumentCommand \foot { } {%
    \tl_use:N \l_first_s_tl
  }
\ExplSyntaxOff
\foot
\end{outputcfb}

\ExplSyntaxOn
\tl_set:Nn \l_first_s_tl { Example }
\tl_set:Nn \l_second_s_tl { ~of~"NV"}
\tl_put_right:NV \l_first_s_tl\l_second_s_tl %correct way to pass the value of variable and not name by using a V-type.
\DeclareDocumentCommand \foot { } {%
  \tl_use:N \l_first_s_tl
}
\ExplSyntaxOff
\begin{nbox2}
\textbf{Output 2:} \\
\foot
\end{nbox2}

\linebreak

\textbf{Example 3}
\begin{outputcfb}
\ExplSyntaxOn
% #1 is argument of "tl_map_inline:Nn" used to insert a single token from the tl var.
\tl_set:Nn \l_first_so_tl { MAPPINGS }%example of mappings
\tl_map_inline:Nn\l_first_so_tl{~it~is~fun~'#1'. \\}
\ExplSyntaxOff
\end{outputcfb}

\begin{nbox2}
\textbf{Output 3:} \\

\ExplSyntaxOn
% #1 is argument of "tl_map_inline:Nn" used to insert a single token from the tl var.
\tl_set:Nn \l_first_so_tl { MAPPINGS }%example of mappings
\tl_map_inline:Nn\l_first_so_tl{~it~is~fun~'#1'. \\}
\ExplSyntaxOff
\end{nbox2}

\linebreak

\textbf{Example 4} \\

\begin{outputcfb}
\ExplSyntaxOn
\tl_set:Nn \l_first_m_tl { { India, } ~ { Singapore, } ~ and ~ { China } }%example of mappings
\tl_map_inline:Nn\l_first_m_tl{'#1',~a~Beautiful~Country~. \\}
\ExplSyntaxOff
\end{outputcfb}

\begin{nbox2}
\textbf{Output 4:} \\
\ExplSyntaxOn
\tl_set:Nn \l_first_m_tl { { India } ~ { Singapore } ~ { China } }%example of mappings
\tl_map_inline:Nn\l_first_m_tl{'#1',~a~Beautiful~Country.~ \\}
\ExplSyntaxOff
\end{nbox2}

\end{document}
