\chapter{The Basics}

2 backticks and 2 single quotes surround a double-quoted string, like ``this''.

\section{Control Sequences}

Also known as \itf{commands}.

2 types: fi{control word} and \itf{control symbol}.

\cil|\TeX| (\TeX) is a \itf{control word} because it is made up of \itf{letters}. Any non-letter will end the sequence.

\cil|\'| (accent on the letter \ttf{o}: \'o) is a \itf{control symbol} because it is made up of a single \itf{non-letter}. The sequence length will always be 1 character (symbol).

To inject a space after a \itf{control word}, do a `\ttf{\bksl \textvisiblespace}'.

\section{Tracing LaTeX}

To trace LaTeX commands, use the \cil|\show| command.

Start a new LaTeX file, say `\ttf{testing.tex}'. Inside, type \cil|\show\thinspace|, and compile. Inside the file \itf{output/testing.log}, you should see:

\begin{outputcfb}
> \thinspace=macro:
->\kern .16667em .
\end{outputcfb}

This means that \cil|\thinspace| simply inserts a space of width \ttf{.16667em}.

Now, you can try \cil|\show\TeX|, which should show:

\begin{outputcfb}
> \TeX=macro:
->T\kern -.1667em\lower .5ex\hbox {E}\kern -.125emX\@.
\end{outputcfb}

This means that \cil|\TeX|:

\begin{itemize}
  \item Shows a \ttf{T},
  \item Reduces the distance between that and the next letter,
  \item Lowers the horizontal box that will be created next,
  \item Creates a horizontal box that contains a letter \ttf{E},
  \item Reduces the distance between that and the next letter,
  \item Shows a \ttf{X}.
\end{itemize}

\subsection{Primitive Control Sequences}

Primitive control sequences will be indicated by the lack of the word ``\itf{macro}''. See \cil|\show\TeX| to remind yourself how a macro definition looks like.

\bff{NB}: Non-primitive control sequences are also known as ``\itf{macros}''.

For example, \cil|\show\kern| will indicate that \cil|\kern| is a primitive:

\begin{outputcfb}
> \kern=\kern.
\end{outputcfb}

Other primitives are \cil|\show|, \cil|\lower|, \cil|\hbox|, etc.

\subsection{Protected Macros}

Protected macros cannot be expanded. To understand that, let's see what happens when you instruct the LaTeX compiler to execute \cil|\show\thinspace|:

\begin{itemize}
  \item Expand \cil|\show|, which executes a function that \bff{displays (not execute) the codes coming next}.
  \item Expand \cil|\thinspace|, which becomes the codes that define the macro \cil|\thinspace|.
\end{itemize}

Normally, the LaTeX compiler will execute the codes that define the macro \cil|\thinspace|. However, since we prefixed that with the command \cil|\show|, the LaTeX compiler simply shows (not executes) the codes.

Now, try to compile \cil|\show\textsuperscript|, and you will see:

\begin{outputcfb}
> \textsuperscript=macro:
->\protect \textsuperscript  .
\end{outputcfb}

Because the \cil|\textsuperscript| is protected, LaTeX refuses to expand it. However, whatever is inside \cil|\csname \endcsname| will always be expanded. The ``\itf{cs}'' here means ``\itf{control sequence}''. \bff{NB}: The argument inside is a control sequence name (eg ``\ttf{textsuperscript}'') without the backslash.

Still, we cannot do \cil|\show\csname textsuperscript \endcsname|. That will only show the definition of \cil|\csname|!

We need to tell LaTex to expand \cil|\show| only after expanding \cil|\csname textsuperscript \endcsname|. So we do

\begin{codecfb}
\expandafter\show\csname textsuperscript \endcsname
\end{codecfb}

LaTeX will compile our instruction in these steps:

\begin{itemize}
  \item Expand \cil|\expandafter| and execute the command to \bff{skip} \cil|\show| \bff{for now}.
  \item Expand \cil|\csname textsuperscript \endcsname| into \cil|\textsuperscript| but with a difference --- that control sequence can now be expanded! (It was protected from expansion, originally.)
  \item Expand \cil|\show|, which executes the command to \bff{show (not execute) the codes of the next command}.
\end{itemize}

\subsection{@ (at) Symbol Restricted}

So,

\begin{codecfb}
\expandafter\show\csname textsupercript \endcsname
\end{codecfb}

gives:

\begin{outputcfb}
> \textsuperscript =macro:
#1->\@textsuperscript {\selectfont #1}.
\end{outputcfb}

That means \cil|\textsuperscript| calls another function --- \cil|\@textsuperscript|.

When we try \cil|\show\@textsuperscript|, we get:

\begin{outputcfb}
> \@=macro:
->\spacefactor \@m {}.
\end{outputcfb}

That means \cil|\@| itself is a \itf{control sequence}! (A \itf{control symbol}, to be precise; recall that a \itf{control symbol} doesn't need a space after.) We need to make the `\ttf{@}' symbol a letter, not a macro. The name of the macro we want to show is `\ttf{@textsuperscript}' (16 letters long).

This is what will work:

\begin{outputcfb}
\makeatletter
\show\@textsuperscript
\makeatother
\end{outputcfb}

And that gives us:

\begin{outputcfb}
> \@textsuperscript=macro:
#1->{\m@th \ensuremath {^{\mbox {\fontsize \sf@size \z@ #1}}}}.
\end{outputcfb}

\section{Macros For Your Convenience}

Non-primitive commands are a combination of primitive and non-primitive commands. As we have seen above, the non-primitive \cil|\textsuperscript| is built with the non-primitives \cil|\@textsuperscript| and \cil|\selectfont|. And digging deeper, the non-primitive \cil|\@textsupercript| is built with these non-primitives --- \cil|\m@th|, \cil|\ensuremath|, and so on.

To see how this all ends in primitives, we look at \cil|\makeatletter\show\m@th\makeatother|, which points us to the primitive \cil|\mathsurround|.

\bff{The point is that all non-primitives are ultimately built with primitives.}

Non-primitives are created for our convenience. For example, to achieve the same effect as \cil|A\textsuperscript\{B\}C| without using the macro \cil|\textsuperscript|, we would need to do:

\begin{outputcfb}
A$\ensuremath{^{\mbox{\fontsize\sf@size\z@ {\selectfont B}}}}$C
\end{outputcfb}

(Note: \cil|\mathsurround| actually gives you \cil|\$ \$|, which creates a \itf{math environment}.)

As you can see in the following, both create the same effect:

\makeatletter
TeX way: A$\ensuremath{^{\mbox{\fontsize\sf@size\z@ {\selectfont B}}}}$C
\makeatother

LaTeX way: A\textsuperscript{B}C

Exercise: Can you try to execute \cil|\TeX| (\TeX) using its underlying definition?

Exercise: What is the underlying definition of \cil|\textsubscript|?
