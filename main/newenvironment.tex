\chapter{New environment}

\section{How to create an environment.}

We have \bff{command} \cil|\newenvironment{name}[args]{begdef}{enddef}| to create new environment.
\begin{itemize}
  \item name: The name of the environment, there must be no currently defined environment by that name and the command name must be undefined.
  \item The environments are delimited by an opening tag \cil|\begin| (\{begdef\}) and a closing tag \cil|\end| (\{enddef\})
  \item args: An integer from 1 to 9 denoting the number of arguments of the newly-defined environment. The default is no arguments.
\end{itemize}

\section{Examples}

\subsection{Create environment without argument}

Defined this code before \ttf{document}
\begin{outputcfb}
\newenvironment{nbox}
  {% Begin codes
    \begin{center}% Show center
    \begin{tabular}{|p{0.7\textwidth}|} % The tabular environment have width 70 percent
    \hline\\ % This will insert a horizontal line on top of the table and at the bottom too
  }
  {
    \\\\\hline
    \end{tabular}
    \end{center}
  }
\end{outputcfb}

Using
\begin{outputcfb}
\begin{nbox}
This is the text formatted by the boxed environment
\end{nbox}
\end{outputcfb}

After build we will have result as below:

\begin{nbox}
This is the text formatted by the boxed environment
\end{nbox}

\subsection{Create environment with argument}

Defined this code before \ttf{document}
\begin{outputcfb}
\newenvironment{nbox}
  {% Begin codes
    \begin{center}% Show center
    \begin{tabular}{|p{0.7\textwidth}|} % The tabular environment have width 70 percent
    \hline\\ % This will insert a horizontal line on top of the table and at the bottom too
    \bff{#1}\\ % Show the argument
  }
  {
    \\\\\hline
    \end{tabular}
    \end{center}
  }
\end{outputcfb}

Using
\begin{outputcfb}
\begin{nboxt}{I'm title}
This is the text formatted by the boxed environment
\end{nboxt}
\end{outputcfb}

Result:

\begin{nboxt}{I'm title}
This is the text formatted by the boxed environment
\end{nboxt}
