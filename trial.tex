\title{Mayuri's prectice page \LaTeXe}
\author{Inclucive}

\documentclass[%
  parskip=full,% empty line after each paragraph.
  twoside=false% not a 2-sided print for a book.
]{scrbook}

\raggedright % keep spaces consistent, easier to see errors in typing.

\usepackage{xcolor} % colors!

\usepackage{listings, tcolorbox}
\usepackage{tcolorbox}
\tcbuselibrary{listings,skins,theorems,xparse}
\NewTotalTCBox{\inlinebox}{ O{white} v}
{ tcbox raise base,
  arc=2pt,
  nobeforeafter,% makes the box inline
  colback=#1!10!white,% pales given color for background
  boxsep=0pt,left=1pt,right=1pt,top=2pt,bottom=2pt,
  boxrule=0pt}
{\lstinline[style=tcblatex,texcsstyle=*\color{blue}\bfseries]§#2§}
\newcommand\texton{\inlinebox[green]}

\NewTotalTCBox{\tunebox}{ O{white} v}
{ colupper=black,nobeforeafter,tcbox raise base,
  arc=5pt,outer arc=0pt,colback=#1!10!white,
  colframe=#1!50!black,
  boxsep=0pt,left=1pt,right=1pt,top=2pt,bottom=2pt,
  boxrule=0pt,bottomrule=1pt,toprule=1pt}
{\lstinline[style=tcblatex,texcsstyle=*\color{pink}\bfseries]§#2§}

\newtcblisting{outputcfb}{
  codefullblock, outputstyle
}


% Shortcuts for oft-used font markup.
\newcommand{\itf}{\textit}
\newcommand{\bff}{\textbf}
\newcommand{\ttf}{\texttt}

% Backslash is used often when writing at LaTeX.
\newcommand{\bksl}{\textbackslash}

% Clickable TOC
\usepackage{hyperref}
\hypersetup{
  colorlinks,
  citecolor=black,
  filecolor=black,
  linkcolor=black,
  urlcolor=black
}

% for multiple copy of text
\usepackage{expl3}
\ExplSyntaxOn
\cs_new:Npn \lrnl3 #1
  { \prg_replicate:nn {#1} {Latex3 \  by \  Mayuri \\*  } }
\ExplSyntaxOff



\usepackage{expl3}
\ExplSyntaxOn
\cs_new:Npn \lrnl3 #1
  { \prg_replicate:nn {#1} {-\  Latex3 \  by \  Mayuri \ ~ } }
\ExplSyntaxOff


\begin{document}

\maketitle

\tableofcontents


\section{A\textsubscript{B}C}

A\textsubscript{B}C %example of textsubscript

\section{latex}
\makeatletter
A$\ensuremath{^{\mbox{\fontsize\sf@size\z@ {\selectfont B}}}}$C
\makeatother

\newcommand{\mayuri}{M \kern-.3667em\lower.5ex\hbox{\textsuperscript{A}}\kern-.1667em Y\kern-.3667em\hbox{\textsubscript{U}}\kern-.1667em R \kern-.3667em\lower.5ex\hbox{\textsuperscript{I}}\kern-.1667em\@}

\mayuri

\makeatletter
M$\ensuremath{^{\mbox{\fontsize\sf@size\z@ {\selectfont A}}}}$Y\kern -.1667em\lower .5ex\hbox {U}\kern -.125emR$\ensuremath{^{\mbox{\fontsize\sf@size\z@ {\selectfont I}}}}$
\makeatother

\section{newcommand}

To create \verb|\newcommand{}| :

\verb|\newcommand{name}[num]{definition}|

two arguments required to make \verb|\newcommand|~~~~~
\{name\}~: command name which you like to use. it should be related to use of command so that easy identified in future and can not be the same name of other command.
[num]: number of arguments, starts from 0.
{definition}: command~ definition~


\newcommand{\lbook}{The book about \LaTeX-learning}


\begin{itemize}
\item \lbook
\end{itemize}

After Install \texton|\tcolorbox and \listings| packages. create \texton|\texton and \newcommand| to show command as text.
Another example of \texton|\newcommand| as bellow.

\tunebox|\LaTeX|
\tunebox[brown]|Some \LaTeX\ code|
\tunebox[blue]{example of new tcolorbox theme command}
\tunebox[red]{\textbf{colorbox fun}}

\subsection{Newcommand with parameters}




\section{environment}

\section{latex3}




Result :

\lrnl3{100}



\end{document}
